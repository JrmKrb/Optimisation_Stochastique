\documentclass{beamer}
\usepackage{lmodern}
\usepackage[frenchb]{babel}
\usepackage[T1]{fontenc}
\usepackage[utf8]{inputenc}
\usepackage{amsmath}

\usetheme{Warsaw}

\title{Maximum Probability Shortest Path Problem}
\author{Jeremy Krebs - Guillaume Soulié}
\institute{Université Paris Saclay}
\date{\today}

\addtobeamertemplate{navigation symbols}{}{%
    \usebeamerfont{footline}%
    \usebeamercolor[fg]{footline}%
    \hspace{1em}%
    \insertframenumber/\inserttotalframenumber
}

\begin{document}

\begin{frame}
\titlepage
\end{frame}

% --------- Sommaire ---------
\begin{frame}
  \tableofcontents
\end{frame}      
% ----------------------------

\begin{frame}
Shortest Path is a known problem and has many applications in "real life".

\begin{itemize}
	\item Goods transport (industrial and private)
	\item Food Delivery (Deliveroo - Foodora - UberEats)
\end{itemize}

\end{frame}

\section{State of the art}

\begin{frame}

As this problem is well known, lots of scientists presented their work on related subject, most of the times with different hypotheses:

\begin{itemize}
	\item<2-> Without resource constraints
	\item<3-> With deterministic resource constraints
	\item<4-> With stochastic resource constraints
\end{itemize}

\uncover<5->{or also with a different optimization problem like utility functions to maximize or cost functions to minimize.}
\end{frame}


\section{Problem}
\subsection{Description}

\begin{frame}
	<Insert Picture from the article and explain orally>
\end{frame}

\subsection{Hypothesis}
\begin{frame}
	\begin{itemize}
		\item Graph with weights on arcs, source node $s$ and sink node $t$,
		\item Stochastic resource consumptions with normal distribution,
		\item K resources,
		\item Threshold C of the cost function (maximum allowed weight of the path).
	\end{itemize}
\end{frame}

\subsection{Formulation}
\begin{frame}

SRCSP can be formulated as this optimization problem:

\begin{align*}
 max\ &\mathbf{Pr} \{ \tilde{a}_k^Tx \leq d_k, k=1..K \} \\ \\
 s.t.\ &c^T x \leq C \\
 &Mx = b \\
 &x \in \{0, 1\}^n
\end{align*}

where:

\begin{itemize}
	\item $x(e) = 1$ if $x(e) \in $ path $P$
	\item The $a_k$ are multi-variate vectors with mean $\mu_k$ and known covariance matrix $V_k$
	\item M is the node-arc incidence matrix. $M(i, e) \in \{-1, 0, 1\}$
	\item b is a vector with 0 everywhere except $b(s) = 1$ and $b(t) = -1$
\end{itemize}

\end{frame}

\begin{frame}

SRCSP can be reformulated as:

\begin{align*}
 max\ & p\\
 s.t.\ & p \leq \mathbf{Pr} \{ \tilde{a}_k^Tx \leq d_k, k=1..K \} \\
 & c^T \leq C \\
 & Mx = b \\
 & x \in \{0, 1\}^n
\end{align*}

\end{frame}

\section{Resolution}
\subsection{One resource case}
\subsubsection{Formulation}
\begin{frame}

With $K=1$ we have:

\begin{align*}
 max\ & p\\
 s.t.\ & p \leq \mathbf{Pr} \{ \tilde{a}_1^Tx \leq d_1 \} \\
 & c^T \leq C \\
 & Mx = b \\
 & x \in \{0, 1\}^n
\end{align*}

\end{frame}

\begin{frame}
Using the known multivariate distribution parameters we get :

\begin{align*}
 max\ & p\\
 s.t.\ & F^{-1}(p)(x^TV_1x)^{\frac{1}{2}} \leq d_1 - \mu_1^Tx \\
 & c^T \leq C \\
 & Mx = b \\
 & x \in \{0,1\}
\end{align*}
\end{frame}

\begin{frame}
Relaxing the problem we get (SRCSPI) with $p \leq \frac{1}{2}$ :

\begin{align*}
 max\ & 0\\
 s.t.\ & F^{-1}(p)(x^TV_1x)^{\frac{1}{2}} \leq d_1 - \mu_1^Tx \\
 & c^T \leq C \\
 & Mx = b \\
 & 0 \leq x \leq 1
\end{align*}
\end{frame}

\subsubsection{Solution}
\begin{frame}
We can solve it using the binary search procedure. We take $p_1 \leq \frac{1}{2}$ a feasible solution, a lower bound of SRCSP. $p_l$ and $p_u$ are lower and upper bounds of SRCSPI. Then we iterate:

\begin{itemize}
\item[Start]<2-> $p_l = \frac{1}{2}$ and $p_u = 1$. Iteration counter $t = 1$
\item[Search]<3-> Solve SRCSPI with $p = p_t$. If SRCSPI has an optimal solution, set $p_l = p_t$, otherwise $p_u = p_t$.
\item[Stop]<4-> Stop when $\frac{p_u - p_l}{2} \leq \epsilon$. Otherwise $t++$ and $p_t = \frac{p_l + p_u}{2} $
\end{itemize}

\end{frame}

\subsection{Joint probabilities}
\begin{frame}
Put the formulation
Explain why we are looking for convexity, using the lectures
Explain how we get the approximation (Theorem 4.1.2), saying that we used this method in classes

\end{frame}

\subsection{Individual relaxed probabilities}
\subsubsection{Formulation}
\begin{frame}
	We formulate this problem taking individual probabilities constraints:
	
\begin{align*}
 max\ & p\\
 s.t.\ & p \leq \mathbf{Pr} \{ \tilde{a}_k^Tx \leq d_k \}\textbf{,\ k=1,..,K} \\
 & c^T \leq C \\
 & Mx = b \\
 & x \in \{0, 1\}^n
\end{align*}
\end{frame}

\begin{frame}

We can relax the equation as we did before to get (RSRCSPJI):

\begin{align*}
 max\ & 0\\
 s.t.\ & F^{-1}(p)(x^TV_kx)^{\frac{1}{2}} \leq d_k - \mu_k^Tx,\ k=1,..,K \\
 & c^T \leq C \\
 & Mx = b \\
 & 0 \leq x_i \leq 1,\ i=1,..,n
\end{align*}
\end{frame}

\subsubsection{Solution}

\begin{frame}

We use the relaxed equation using the Binary Search Procedure again:

\begin{itemize}
\item[Start] $p_l = \frac{1}{2}$ and $p_u = 1$. Iteration counter $t = 1$
\item[Search] Solve RSRCSPJI with $p = p_t$. If RSRCSPJI has an optimal solution, set $p_l = p_t$, otherwise $p_u = p_t$.
\item[Stop] Stop when $\frac{p_u - p_l}{2} \leq \epsilon$. Otherwise $t++$ and $p_t = \frac{p_l + p_u}{2} $
\end{itemize}
\end{frame}

\subsection{Results}
\begin{frame}
Show results and analyze them (Why is this algorithm faster, why not, etc)
\end{frame}

\end{document}
